\documentclass[journal]{IEEEtran}
\IEEEoverridecommandlockouts
\usepackage{fancyhdr}
\usepackage{graphicx}
\usepackage[spanish, es-tabla]{babel}
\usepackage[utf8]{inputenc}
\usepackage{color}
\usepackage{hyperref}
\usepackage{wrapfig}
\usepackage{array}
\usepackage{multirow}
\usepackage{adjustbox}
\usepackage{nccmath}
\usepackage{subfigure}
\usepackage{amsfonts,latexsym} % para tener disponibilidad de diversos simbolos
\usepackage{enumerate}
\usepackage{booktabs}
\usepackage{float}
\usepackage{threeparttable}
\usepackage{array,colortbl}
\usepackage{ifpdf}
\usepackage{rotating}
\usepackage{cite}
\usepackage{stfloats}
\usepackage{url}
\usepackage{listings}
\usepackage{amsmath}
\usepackage{makecell}%To keep spacing of text in tables
\setcellgapes{2pt}%parameter for the spacing in tables

\newcolumntype{P}[1]{>{\centering\arraybackslash}p{#1}}  %% Se crea un nuevo tipo de columna llamada P.
\newcommand{\tabitem}{~~\llap{\textbullet}~~}
\newcommand{\ctt}{\centering\scriptsize\textbf} %%\ctt abrevia el comando \centering\scriptsize\textbf
\newcommand{\dtt}{\scriptsize\textbf} %%\dtt abrevia el comando \scriptsize\textbf
\renewcommand\IEEEkeywordsname{Palabras clave}



\hyphenation{} 

\graphicspath{ {figs/} } 



\newcommand{\MYhead}{\smash{\scriptsize
\hfil\parbox[t][\height][t]{\textwidth}{\centering
\begin{picture}(0,0) \put(-0,-17){\includegraphics[width=33mm]{LogoUMNG}} \end{picture} \hspace{5.9cm}
INFORME DE PRÁCTICA DE LABORATORIO \hspace{4.7cm} Versión 1.0\\
\hspace{6.45cm} PROGRAMA DE INGENIERÍA MECATRÓNICA \hspace{4.65cm} Periodo 2022-1\\
\underline{\hspace{ \textwidth}}}\hfil\hbox{}}}
\makeatletter
% normal pages
\def\ps@headings{%
\def\@oddhead{\MYhead}%
\def\@evenhead{\MYhead}}%
% title page
\def\ps@IEEEtitlepagestyle{%
\def\@oddhead{\MYhead}%
\def\@evenhead{\MYhead}}%
\makeatother
% make changes take effect
\pagestyle{headings}
% adjust as needed
\addtolength{\footskip}{0\baselineskip}
\addtolength{\textheight}{-1\baselineskip}

%%%%%%%%%%%%%%%%%%%%%%%%%%%%%%%%
\begin{document}

\title{Reconocimiento de caracteres con redes neuronales}

\author{Julián~Garzón,~
        Alejandro~Martínez,~Yery~Pedraza,~Oscar~Rodríguez,~
        y~Santiago~Téllez\\
				\textit{est.\{,~yery.pedraza,~oscar.rodriguez9,~y~\}@unimilitar.edu.co}\\
				Profesor:~Camilo~Hurtado\\% stops a space
\thanks{El presente documento corresponde a un informe de desarrollo práctica de laboratorio de Inteligencia Artificial presentado en la Universidad Militar Nueva Granada durante el periodo 2022-1.}} %\thanks anexa una nota a pie de página donde se puede colocar alguna información sobre la naturaleza del documento.
%%%%%%%%%%%%%%%%%%%%%%%%%%%

% Comando que indica la generación del título
\maketitle

\begin{abstract}
Breve resumen de la tarea, los ejercicios realizados y algunos resultados obtenidos.
\end{abstract}

%%%%%%%%%%%%%%%%%%%%%%
\begin{IEEEkeywords}
Escribir las palabras clave que se encuentran en el documento.
\end{IEEEkeywords}
%%%%%%%%%%%%%%%%%%%%%%
%\IEEEpeerreviewmaketitle


\section{Competencias a desarrollar}
\begin{itemize}
\item Competencia 1.
\item Competencia 2.
\item Competencia 3.
\end{itemize}


\section{Desarrollo de la práctica}
En esta sección se describen los diferentes experimentos, cálculos y simulaciones solicitados, junto a los resultados obtenidos representados mediante gráficas y tablas.

\subsection{Descripción de patrones a reconocer}

\subsection{Implementación del modelo manual}

\subsection{Implementación del modelo con Python y Keras}

\subsection{Comparación y elección entre los métodos propuestos}

\subsection{Ajuste del modelo}

\subsection{Implementación de visión de máquina}

\subsection{Implementación de interfaz gráfica de usuario}

\section{Conclusiones}


\ifCLASSOPTIONcaptionsoff
  \newpage
\fi


\begin{thebibliography}{1}

\bibitem{imagenes}
Youtube, canal schaparro. \url{https://youtu.be/IhvF6iY7n5k}. Recuperado el 30 de Enero de 2017.

\bibitem{GNUp}
Gnuplot homepage. \url{http://www.gnuplot.info/}. Recuperado el 26 de Julio de 2018.

\bibitem{dia}
Dia Diagram Editor. \url{https://sourceforge.net/projects/dia-installer/}. Recuperado el 30 de Enero de 2017.

\bibitem{xcircuit}
XCircuit, Open Circuit Design. \url{http://opencircuitdesign.com/xcircuit/}. Recuperado el 26 de Julio de 2018.

\bibitem{nombre_para_citar}
Inicial1.~Apellido1 and Inicial2.~Apellido2, \emph{Nombre de libro}, \#edición~ed.\hskip 1em plus
  0.5em minus 0.4em\relax Ciudad, País: Editorial, año.

\bibitem{kopka}
H.~Kopka and P.~W. Daly, \emph{A Guide to \LaTeX}, 3rd~ed.\hskip 1em plus
  0.5em minus 0.4em\relax Harlow, England: Addison-Wesley, 1999.

\bibitem{link}
Overleaf. \url{https://www.overleaf.com/}. Recuperado el 02 de Febrero de 2017.

\end{thebibliography}
%%%%%%%%%%%%%%%%%%%%%%%%%%

\end{document}
%%%%%%%%%%%%%%%%%%%%%%%%%%%%%%%%
%%%%%% FIN DEL DOCUMENTO %%%%%%%
%%%%%%%%%%%%%%%%%%%%%%%%%%%%%%%%




